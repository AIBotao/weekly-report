\documentclass[11pt]{report}

\special{papersize=8.5in,11in}

\topmargin -0.5in \oddsidemargin 0.00in \evensidemargin 0.00in
\textwidth 6.75in \textheight 9.0in \headheight 0.25in \headsep
0.25in \footskip 0.5in \hoffset 0in \marginparpush 0.0in
\marginparwidth 0.0in \marginparsep 0.2in

\setcounter{page}{1}

\newcommand{\D}{\displaystyle}\newcommand{\T}{\textstyle}
\newcommand{\e}{{\mathrm{exp}}}
\newcommand{\dd}{{\mathrm d}}
\newcommand{\comment}[1]{}
\newcommand{\mb}{\mathbf}
\reversemarginpar

\usepackage[final]{graphicx}
\usepackage{fancyhdr}
%\graphicspath{{Papers/}}
\usepackage{amsthm,amssymb,amsmath}
\usepackage{cite}
\usepackage{geometry}
\usepackage{amsmath}
\usepackage{booktabs}
\usepackage{color}
\usepackage{setspace}
\usepackage{subfigure}
\usepackage{url}
%\usepackage[top=2.5cm, bottom=2.5cm, right=3.5cm, left=3.5cm]{geometry}
\geometry{a4paper,scale=0.8}
\setcounter{secnumdepth}{3}

\title{Research Progress Report}

\author{Botao Zhu}

\begin{document}
	
	\maketitle
	\lhead{\sf Research Progress Report} \chead{} \rhead{\sf Botao Zhu}
	\lfoot{CTRG, University of Saskatchewan} \cfoot{} \rfoot{Page \thepage}
	\renewcommand{\footrulewidth}{1.0pt}
	\renewcommand{\headrulewidth}{2.0pt}
	\renewcommand{\arraystretch}{1.3}
	\pagestyle{fancy}
	
	\renewcommand{\thesection}{\arabic{section}}
	
	\section{Reading and Research Activities}
	
	\subsection{Summary}
	
	\noindent\cite{5408367} 

	
	\noindent However, because
	
	\subsection{Deep Learning Based Routing Strategy}
	\subsubsection{Input and Output Design}
	
	\subsubsection{Deep Learning Structure Design}
	The authors choose the DBA as the deep learning structures as shown in Figure~\ref{1stfig}.
	\begin{figure}[h!]
		\centering
		%\includegraphics[width=0.5\linewidth]{figure1.png}
		\caption{Considered L-layer DBA}
		\label{1stfig}
	\end{figure}
	
	\subsubsection{The Procedures of the Proposed Deep Learning Based Routing Strategy}
	\paragraph{Initialization Phase}

	\paragraph{Training Phase}
     
	\begin{tabular}{lc}
		\toprule
		\textbf{Algorithm 1}. Supervised Train DBA\\
		\hline
		\textbf{Input:} $\left(x,y\right)=\{\left(x^t,y^t\right)|t=1,...,m\}, \eta_{CD}, \eta_{bp},L, n=\left(n_1,...,n_L\right)$\\
		\textbf{Output: $\theta$}\\
		1: \textbf{for} $i=1,...,L-2$ \textbf{do}\\
		2: \quad TrainRBM $\left(u^i,\eta_{CD},n_i,n_{i+1}\right)$\\
		3: \textbf{end for}\\
		4: Fine-tuneDBA$\left(\left(x,y\right),\theta,\eta_{bp}\right)$\\
		5: \textbf{return} $\theta$\\
		\hline
	\end{tabular}

	\paragraph{Running Phase} 

	
	\begin{table}[!h]
		\centering
		\caption{Routing table of $R_3$}
		\begin{tabular}{lc}
			\toprule
			Dest& Path\\
			\hline
			$R_1$& $R_3 \to R_2 \to R_1$\\
			$R_2$& $R_3 \to R_2$\\
			\dots& \dots\\
			$R_{16}$& $R_3 \to R_7 \to R_{11} \to R_{15} \to R_{16}$\\
			\hline
		\end{tabular}
	\end{table}

	
	\subsubsection{Network Performance Analysis}
	They compared the proposed deep learning method with OSPF by the network signaling overhead, throughput and average delay. 
	\begin{figure}[!htbp]
		%\centering
		\subfigure[Comparison of signaling overhead]{
			\begin{minipage}[t]{0.35\linewidth}
				\centering
				\includegraphics[width=2.4in]{figure5}
			\end{minipage}%
		}%
		\subfigure[Comparison of throughput]{
			\begin{minipage}[t]{0.35\linewidth}
				\centering
				\includegraphics[width=2.4in]{figure6}
			\end{minipage}%
		}%
		\subfigure[Comparison of average delay]{
			\begin{minipage}[t]{0.35\linewidth}
				\centering
				\includegraphics[width=2.4in]{figure7}
			\end{minipage}
		}%
	\centering
	\caption{Comparison of network performance under different network loads}
	\end{figure}
	
	\section{Objectives for the Next 2 Weeks}
	\subsection{Reading} 
	Reading papers foused on ML-based or DL-based routing.
	\subsection{Course} 
	Studying chapter 1 and chapter 2 of Neural Networks and Deep Learning, \textbf{Coursera}. \url{https://www.coursera.org/learn/neural-networks-deep-learning}
	\subsection{Code}
	Studying the classic routing protocol: LEACH and using Matlab to implement.
	
	\section{Advisor's Comments}
	
	\bibliographystyle{IEEEtran}
	\bibliography{janbib}
	
\end{document}